\chapter{}
\section*{Из воспоминаний Н.Б. Левданской (Канатрович)}

Я родилась  в  1923 году  в  Москве. Наша  семья  занимала  две  комнаты 
коммунальной  квартиры  в  доме  №21  на Кузнецком мосту,  там,  где  памятник 
Воровскому.  Два  подъезда  этого жилого  дома~-- №№  6 и 7~-- занимал Наркомат 
иностранных дел, где служили мои родители.

Мама~-- Бетти  Моисеевна  Канторович  (Маркус)  была  машинисткой, а отца~-- 
Бориса  Ильича  Канторовича~-- приняли  на работу по рекомендательному  письму 
Ленина Чичерину. Дело в том, что папа после того, как окончил гимназию с  золотой 
медалью  в Гомеле,  где жила  его  большая семья (у бабушки Розы было 4 сына и 3 
дочери), поступил в университет  г. Льежа (Бельгия),  кстати, самый  дешевый  в  то 
время  в  Европе.  Туда  приезжал  из  Швейцарии  Ленин,  который  вел  беседы  со 
студентами  Русского  землячества.  Вернулся  в  Москву  папа  в  том  же 
запломбированном ленинском вагоне. Попал  он туда благодаря своему  маленькому 
росту, как подросток~-- сын одной из уезжавших семей.

Отец  был одним  из организаторов структуры Наркоминдела,  наших  посольств 
в других  странах.  По работе он занимал разные должности,  был  членом  коллегии 
Наркомата.

В конце  20-ых годов, когда  Наркоминдел  стал  расширяться  и  должен  был 
занять весь дом,  а жильцы расселены, встал вопрос о строительстве кооперативного 
дома, что и было сделано  совместно с другими организациями.  В апреле  1929  года 
мы переехали в дом № 2/6  по Хоромному тупику  в пятикомнатную квартиру № 64. 
К нам с Арбата переселилась бабушка Таня, мать мамы.  В этот же дом переселилась 
сестра мамы Елена Моисеевна Маркус  с  мужем  Леоном  Яковлевичем Гайкисом, 
тоже наркоминдельцы.

После коммуналки дом был  воплощением мечты: отдельная квартира,  ванная, 
газовая  плита. Для  детей  были открыты детский сад  в  квартире  первого  этажа 
в подъезде № 2, замечательный Красный уголок в подвале  5-го подъезда,  где  были 
библиотека, кинопередвижка,  разные  кружки. Для  хозяек к дому был  прикреплен 
гастроном  № 3: по телефону можно было заказать продукты, а  после работы  взять 
их в специальных ячейках, сделанных в стене входной арки.

В  1930 г.,  в связи с  индустриализацией страны, было указание  о переводе  в 
промышленность  работников  с  техническим  образованием.  А так  как  папа был 
инженером по строительству дорог и мостов, он был  направлен в ЦУДортранс и два 
года работал  по  реконструкции  Военно-Грузинской дороги. Мама же  после НКИД 
поступила во Внешторгбанк на должность иностранного корреспондента, т.к.  знала 
несколько языков.

В эти же годы родители вступили в дачный кооператив  и  получили  маленькую 
фанерную дачку близ платформы Валентиновка Северной ж.д.

Период  1935-36 гг  был  тяжелым.  ЦУДортранс  ликвидировали,  т.к.  его 
руководство проходило по процессам,  работы  не было, материально было трудно. 
Немного  помогал Торгсин,  куда  отнесли золотые  медали  родителей,  кое-какие 
бабушкины вещи.  В конце концов  папе  удалось  устроиться  в  плановый отдел 
Мострамвайтреста.

Л.Я. Гайкис  продолжал работать в Наркоминделе  и  в 1935 г. был  направлен 
генконсулом посольства в Турции, куда отправились  всей семьей: тетя Лена  и  две 
их дочки:  Наташа, родившаяся в  1930 г., и Оля  1934 года рождения.  Свою квартиру 
они  передали Е.В. Голубевой, вдове  И.А. Дивильковского, которого знали по НКИД. 
Гайкису  в дальнейшем  была обещана  квартира  в новом кооперативном доме  на 
Каляевской.  В начале  1937 г. Леон Яковлевич  получил  назначение  на  должность 
посла  в Испании  и отправился  в Мадрид,  а семья  из-за  войны  поселилась  во 
Франции, где тетя Лена стала работать в нашем посольстве.

Наступил страшный  1937 год. Каждую  ночь  люди  НКВД  на черных  <<эмках>>
увозили арестованных.  23 мая  эти  люди  пришли к нам. Был  обыск, и на рассвете 
папу увели, оставив  нам номер  ордера  и опечатав  все  пять комнат. Маму после 
этого  парализовало,  мы  с  домработницей  Маней  вывезли  ее  кровать в  коридор. 
Мама лежала без медицинской помощи.

Мы  остались  без  самых необходимых  вещей. Предстояла зима,  а  у  нас  вся 
теплая одежда оставалась в запертых комнатах. Я пошла к  прокурору.  Он сказал: 
<<Если хотите остаться живыми, уезжайте из Москв>>.  И  дал  указание  открыть 
на время  одну-две  комнаты, чтобы  мы  могли  взять  необходимые  вещи.  К  нам 
пришел человек, и под его надзором  мы с  Маней  взяли  самое необходимое. Один 
раз мне удалось проникнуть в  опечатанную  комнату без  надзора. У нас с соседями 
был общий балкон, перегороженный сеткой, я  с их разрешения через нее перелезла, 
с помощью  форточки  открыла  окно,  влезла  в  бабушкину  комнату  и  взяла  из 
буфета серебряные  ложки и подстаканники.  Слава  Богу, что  бабушка  не  видела 
всего этого: ее не стало в феврале  1936 года.

В школе я сдала экзамены за шестой класс, и  мне помогли  перевезти  маму  на 
нашу  дачу в Валентиновке.  Отношение  людей  к нам  было  разное:  кто  помогал, 
кто сочувствовал, а кто при виде меня переходил на другую сторону улицы.

В этом  1937 году в Харькове и  Киеве  арестовали  трех  братьев  отца  вместе  с 
женами.  Мужчин  расстреляли,  женщины,  кроме  моей  мамы,  вернулись,  но 
инвалидами.

Л.Я. Гайкиса  тоже  не  миновала  эта  участь:  в  июле 1937 г.  его  отозвали  в 
Москву  и арестовали в гостинице «Москва», где он остановился. Тетю Лену  тут же 
уволили с работы и отправили в Москву, куда она с девочками  и  прибыла. Но  жить 
было негде; тогда  Лена вместе  с Ниной  Исидоровной  Мельниковой (с той же судьбой)
сняли комнату в  деревне на станции Калистово в Подмосковье.

В домоуправлении мне выдали справку, с этим документом  я ходила  по  разным 
инстанциям. О папе можно  было  узнать в справочной  НКВД  на  Малой Лубянке. 
Очереди были огромные. Не  всегда подойти  к окну удавалось  за  весь день,  тогда 
мы оставались  на следующий.  На  этой улице  тогда  был  небольшой  костел,  туда 
пускали на ночь.  Каких  трагедий  я  навидалась  в  этих  очередях~--- не передать... 
Подойдя к окну,  надо  было  назвать  номер  ордера, фамилию  и дать  3 рубля  ( на 
курево заключенному); если деньги брали,  это  значило, что  человек жив.  У меня 
там взяли  в июне  и в июле, а  в августе  мне это удалось  в Бутырках,  т. к.  Малую 
Лубянку закрыли.

Когда  в сентябре  отняли  и  нашу  маленькую  дачку,  наша  домработница 
Маня  предложила нам поехать  на ее родину  в  город Углич.  Маня  съездила туда, 
нашла нам комнату  и  потом  приехала  за  нами.  В  30-е  годы  в Угличе строили 
гидроузел, и  там  провели  железную  дорогу  от  Калязина  до  Углича (40 км). 
Доехав из Москвы  до Калязина, мы перебрались в вагон-теплушку,  поставили там
мамину  кровать, так  добрались  до Углича и  сняли  комнату,  которую  Маня нам 
заранее подготовила. В Угличе я  начала  учиться  в седьмом  классе  и работать  в 
школе, а  мама, когда  поправилась,  поступила  на  работу  счетоводом  в  артель 
инвалидов.

Каждые каникулы я ездила в Москву, ходила  в справочное  НКВД  на Кузнецком 
мосту,  в  прокуратуру  на  Пушкинской  улице,  и  везде  получала  ответ:  «10 лет 
строгого режима без права переписки».  Тогда мы  еще  не  знали, что это означает~-- 
расстрел.  Когда  через  много лет, в 1992 году, я  прочла  дело отца, я узнала, что 31 
октября 1937 г. был суд, продолжавшийся  18 минут, а 1  ноября отец был  расстрелян.

В  1939 г. в деревне  Калистово  была  арестована  Н.И. Мельникова,  тогда  мы 
решили  поскорее забрать тетю Лену с девочками  в Углич.

В  1939 -- 40-м годах  я  ездила на летние каникулы  в  Харьков, где папина сестра 
собирала  на  отдых  нас, племянников-сирот.  Там  я  познакомилась  с  прекрасным 
парнем Наумом Соколовским, семья  которого сыграла большую роль  в моей  жизни. 
Мы  долго переписывались, а 21  июня 1941  года, когда я окончила школу, он приехал 
в Углич, чтобы забрать меня в Харьков, где я могла бы продолжить свое образование. 
Но на другой день началась война, и он сразу уехал, т.к. дома у него  была повестка в 
военкомат. Забегая вперед, скажу, что  он достойно воевал  и погиб  в  1943 году  под 
Гомелем. В  1967 г., получив  известие от  его матери, которая  после долгих поисков 
узнала место захоронения сына, я приехала в деревню  Барсуки, где  встретили меня <<красные следопыт>>~-- школьники 5~-- 6 классов, которые занимались поисками мест 
захоронения. Меня проводили в лучший по тем временам дом,  где  меня  с  большой 
теплотой приняла хозяйка  и подарила  мне кусочек драгоценного  для того  времени 
мыла. Но как велико было мое изумление  и моя радость, когда она  разыскала  среди 
многих  полуистлевших бумажек мое письмо, написанное  моему  жениху!  Красные 
следопыты  вместе с односельчанами  привели меня к братской  могиле, на  которую 
я  положила цветы.

В Угличе на второй  день войны, 23 июня, меня послали  от  горкома комсомола 
проводить собрания в колхозах. Когда я возвращалась домой, меня встретили друзья- одноклассники  и  сообщили, что  маму  увезли  в  больницу.  Но, когда  я  увидела 
опечатанную дверь комнаты, я поняла, что случилось, и бросилась  к дому, где  жила 
тетя  Лена  с  детьми. Там  в это  время  шел  обыск,  меня  не  пустили,  попросту 
выкинули в коридор. Детей на это время  приютили  соседи. Я пошла в отдел НКВД, 
комнату мне вернули и разрешили девочек  взять к себе, т.к. мне уже исполнилось 18 
лет.

Потянулись  страшные  военные  дни. Осенью Наташа пошла  в школу, а Оля~-- в 
детский  сад. Было холодно (дрова на рынке  стоили дорого) и голодно,  ведь нам  не 
давали хлебных карточек, как детям «врагов народа». Правда, в  горкоме  комсомола 
в буфете раз  в неделю мне давали буханку хлеба. Ходила по деревням, меняла вещи 
Лены (своих не было) на муку,  на картошку. Работала я в школе,  летом  с  ребятами 
в  поле помогали колхозу, а осенью нас, взрослых,  возили  на  рытье  окопов,  ведь 
немцы  были близко.

Пыталась найти  маму и Лену. Посоветовали писать в отделы НКВД ближайших 
областей. Осенью ответили из Ярославля. Добралась я туда  с трудом, т. к.  никакого 
регулярного  сообщения  не было,  а попутные грузовики  брали  только  за  махорку 
или водку. Там  я узнала,  что  мама  и  Лена  вместе, следователь  уговаривал  меня 
сдать девочек в детдом, но об этом я даже думать не могла.

Так прошел первый тяжелый военный год.  Наум Соколовский  регулярно  писал 
с фронта,  и  с его родителями я тоже переписывалась.  Они были  в  армии, но  не  в 
строевых частях. К маме и Лене ездила несколько раз, отвезла им теплые вещи.

Летом  1942 года за нами приехал отец Наума Иосиф Яковлевич и отвез меня  с 
девочками в город  Рыбинск, где  расположилась  военная база, и мы стали  жить в 
этой  гостеприимной  семье.  Иосиф  Яковлевич имел звание капитана, а  жена его~-- 
Евгения Наумовна~-- была  вольнонаемной.  Я  опять  работала  в  школе, девочки 
учились.  Так  получилось,  что  лагерь, где  были  мама и Лена,  тоже оказался  в 
Рыбинске.  Я  могла  иногда издали  наблюдать, как  их этапом  ведут  по городу от 
зоны  к  Волге,  и  с  ужасом  видеть, что  моя  мама, инвалид,  по  колено  в  воде 
вылавливает  плывущие доски. Единственное, что  мне  удавалось,~-- это  передавать 
им через проходную  кое-какую еду.

Встал  вопрос  о моей  дальнейшей  учебе.  Дважды  я  посылала  в  разные 
московские вузы свои документы, и  все они, несмотря  на <<золотой>>  школьный 
аттестат,  возвращались  обратно.  Тогда  моя  двоюродная  сестра,  студентка 
строительного  института,  придумала  мне  подходящую  биографию  и  вместе с 
аттестатом  отнесла  в  МИСИ.  И  ведь  приняли!  Соколовские  отпустили  меня, 
оставив  девочек  у себя,  и осенью  1943 года я оказалась в Москве.

Прописали меня в общежитии на Извозной улице, но там  было  трудно жить  и 
заниматься. В здание попала во время налетов бомба, его еще  не отремонтировали, 
окна были забиты  фанерой, вместо лестничных маршей лежали доски, поэтому там 
я  жила не постоянно, а <<скиталась>>  по городу, пользуясь гостеприимством  друзей 
и  знакомых.

В  1944 году  нам  с  двоюродной  сестрой,  студенткой-медичкой,  дальние 
родственники дали  возможность временно  пожить  в пустующей  комнате (хозяин 
был на фронте) коммунальной  квартиры  на  Домниковке.  Хотя  было холодно  и 
голодно, мы были  рады. Подружились  с  соседями.  Там же  я  познакомилась  с 
моим  будущим мужем  Максимом  Ильичом  Левданским.  Он  воевал,  был  ранен 
и  контужен  под Сталинградом,  прошел  несколько госпиталей,  последний  был 
в  Москве.  Нашему  соседу он был другом детства.

Девочки  оставались  у  Соколовских.  Из  Рыбинска  военную  базу перевели  в 
Можайск, куда я ездила по выходным, а потом в город Ржев. Лагерь, где были мама 
и Лена, тоже  <<переезжал>>: война  отходила  на запад, нужна была рабочая сила для 
восстановления  разрушенных дорог,  домов  и  прочего  хозяйства.

В январе  1945 г.  неожиданно  тетю  Лену  из  лагеря,  который  в это  время был 
в Уваровке (недалеко от Бородино) привезли  в Москву.  Освободила её организация 
<<Польские патриоты>>, которая искала людей для освобожденной от немцев родины, 
а фамилия  Гайкис  была известна  в  Польше. Жить  было негде, но есть  хорошие 
люди: Лену и девочек, которых  Соколовские привезли  из Ржева, взяли  к  себе  в 
одну комнату коммуналки  старые друзья.  В июне  1945 г. был сформирован  целый 
эшелон  из  людей, в основном, найденных в Гулаге, и отправлен в Варшаву.

А  я в сентябре  1945 года  вышла  замуж  за  М.И. Левданского  и  поселилась  в 
Томилино, где он жил с мамой. Наконец я обрела крышу над головой и нормальную 
жизнь. У нас  была половина дома (две комнаты и терраса  с  удобствами на улице), 
но все равно  это счастье.

В  1946 г. у меня родилась дочь Таня,  в  1949 г.~-- сын Илюша.

Доучиться  в  МИСИ  не  получилось,  свое  образование  я  заканчивала  через 
несколько  лет  заочно.  Надо  было  растить  детей,  к тому  же  свекровь  серьезно 
заболела и в августе  1948 г. скончалась. По страшному совпадению  в этот  же  день 
не  стало  моей  мамы.  Из  лагеря пришла  обратно посылка с наклеенной бумажкой. 
Случилось  это в Горьковской области. Маме было 53 года.

В  1951  г. с помощью моих сокурсников, ставших уже инженерами, я  поступила 
в  проектный институт, в котором проработала до пенсии, постепенно  <<поднимаясьпо  служебной  лестнице>>,  дойдя  до  должности  главного  инженере  проекта.
Обстановка  была рабочая, коллектив прекрасный (общаемся до сих пор).  Я  много 
ездила в командировки~--- и  на выбор площадок под строительство, и  на  авторский 
надзор.

Соколовские  после  войны  вернулись  в Харьков. Мы все~-- и я, и муж, и  дети~--
ездили к ним, один раз вместе с Олей,  которая приехала из Варшавы, помогали  им 
продуктами, лекарствами, деньгами вплоть до их кончины.

Жизнь в Томилино шла своим чередом. Дом постепенно улучшался:  пристроили 
кухню,  утеплили  террасу. На  участке  росли  вишни,  малина, смородина; был 
небольшой  огород,  цветы.  Но все  это  рухнуло,  когда  а  октябре  1963 года  от 
обширного  инфаркта  скоропостижно,  во сне. скончался  муж.  Случилось  это  в 
Крыму, в доме отдыха. Максиму Ильичу было 5O лет.

Осиротели  мы  с ребятами. Жить  стало трудно, пенсию  на  детей  по  утрате 
кормильца назначили~-- 38 руб.43 коп.  Хваталась  за  любые  подработки,  дети 
помогали. Они у меня замечательные,  школу окончили  оба  с  золотой  медалью  и 
поступили в институты.

После  смерти  мужа  тяжело  было  содержать  наш  дом. и потребовался год 
неимоверных усилий, чтобы обменять его на небольшую  двухкомнатную квартиру 
с удобствами. Активно помогали нам с хлопотами сослуживцы мужа и  сотрудиики 
моего  института.  Так  мы  стали  жить  в  пселке  Института  горного дела  им. 
Скочинского, это рядом с Томилино.

Шли годы, дети повзрослели, создали свои семьи:  в 1968 г. Таня  вышла замуж 
за  своего  одноклассника  Костю,  а  в  1970 г.  Илья  женился  тоже  на  своей 
однокласснице Миле. Все  четверо окончили  вузы, стали  работать. И зять и  сын 
отслужили  в  армии. Родились внуки.  И  я решилась в 1975 г.  вторично  выйти 
замуж.  С  Михаилом  Ивановичем  Касьяновым  я  проработала  вместе  более  20 
лет. Он в свое время прошел ГУЛАГ, овдовел,  был  на  пенсии.  В его семье  тоже 
были дети, внуки, причем дочь замужем за М. М. Плоткиным  (он из нашего  дома  в 
Хоромном), тоже сыном <<врага народа>>.

В 1978 г. я вышла на пенсию,  пришло время  помочь  детям.  Четверо  внуков 
пошли в детский сад, нужно было их приодеть. Купить в те годы было нечего,  и  я 
стала  портнихой.  Из  взрослых  ношеных  вешей  перешивала  внукам  платьица, 
юбочки, брючки и проч.

С опытом  дело  пошло  лучше,  и  вскоре  я  стала  <<обслуживать>>  и  своих 
взрослых детей. Мне всегда хотелось научиться шить, но не было времени,  к тому 
же  с  уходом  на  пенсию я получила от сослуживцев в подарок швейную  машинку 
с  электроприводом.  Она  и  сейчас  выручает, правда,  нового  почти  не  шью, в 
основном, заплаты и переделки.

С  Михаилом Ивановичем мы прожили  почти  13 лет. За  эти  годы побывали с 
ним, кроме  Подмосковья,  в разных местах: в Карпатах, в Одессе и Черновцах,  на 
Валааме. Первое  время  мы  жили  с семьей  его сына  в доме  у Чистых  прудов, а 
затем  нам  дали  однокомнатную квартиру в доме на ул. Новая дорога, где  я живу 
до сих пор. Так снова я стала  москвичкой,  да еше  и  в родном  Бауманском (ныне 
Басманном) районе.

Уже 2S лет, как  нет  Михаила  Ивановича;  с  его  семьей  у  меня  прекрасные 
отношения,  мы  перезваниваемся,  встречаемся,  вместе отмечаем разные  события.

Жизнь  так повернулась,  что  я  получила многое,  что было отнято.  У  моих 
детей  большие  семьи.  У дочки  Татьяны  Максимовны  с мужем  Константином 
Анатольевичем  сын Сергей,  у которого с женой Катей  двое детей, и  дочь Ольга  с мужем  и  сынишкой. У сына  Ильи Максимовича с женой  Людмилой  Евгеньевной 
две дочки~--  Маша и Наташа,  два  зятя~--  Антонио  и  Александр  и четверо внуков.

Все  мои  дети  и  внуки получили  высшее образование,  все  работают  (кроме 
дочки и невестки, которые вышли на пенсию и воспитывают своих  внуков).

Мое  продолжение~-- правнуки.  Старший~-- Саша~--  пойдет  в  восьмой  класс, 
две девочки~-- Вероника и Соня~-- в четвертый,  два  мальчика~-- Максим  и  Миша~-- 
первоклашки, и двое маленьких, ясельных,~--  Петя и Бейка (Беатриче).

Вот  такая  гвардия!  Если же  прибавить внуков и правнуков  Касьяновых,  то 
получается огромное число.

Я живу  одна,  но  одинокой  себя  не считаю. Мои  частенько  навещают, и  я 
езжу  к  ним,  стараюсь  хоть  чем-то помочь,  делаю  все  дела по  дому,  читаю, 
смотрю  телевизор  (в основном, «Культуру»),  стараюсь  не  отстать  от  жизни. 
Конечно, бывают  и  грустные  мысли:  самая  большая  боль~--  что  нет  могил 
родителей, а еще ~-- нездоровье, но я не сдаюсь.

Оглядываясь  назад, вижу, что в  моей  долгой, почти  девяностолетней  жизни 
было  много  страшного,  трагического,  но,  к  счастью,  и  немало  хорошего. 
Встречались прекрасные люди, понимающие, помогающие. Вот так и выстояла.

В  1957 г.  после  XX  Съезда  все  семьи  Канторович  и  Гайкис  были 
реабилитированы.

Хочу  немного  добавить  о семье  Гайкис.  В Варшаве Лена пошла работать, а 
девочки учились. Все  они приезжали в Москву и хотели,  чтобы  и  я навестила их, 
прислали  приглашение, но  ОВИР отказал  с  формулировкой:  <<Не прямая родня>>. 
Было очень обидно.

После школы девочки окончили университет, создали  семьи,  родились  дети~-- 
и все  кончилось, когда  Гомулка выслал  из  Польши  всех евреев, буквально в 24 
часа.

Лена и  семья  Наташи  перебрались  в Израиль. Там вскоре ушли из жизни~--  и 
Наташа (ей было 43 года), и тетя Лена.

Оля с мужем и дочкой уехали в  Америку,  где  получили  работу в Йельском 
университете  на кафедре  русского языка  и литературы.  Оля проработала там 30 
лет, сейчас на пенсии. После смерти мужа перебралась  в Нью-Йорк, а дочка ее~--  в 
Вашингтоне с мужем и двумя девочками-студентками,  хорошими спортсменками. 
Оля с мужем много раз приезжали ко мне, сейчас регулярно перезваниваемся.