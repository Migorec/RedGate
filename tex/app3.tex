\chapter{}

\section*{Из воспоминаний М.В. Миникса о хозяйстве и работниках Дома у Красных ворот\protect\footnote{Пишу все, что вспомнил и как помню. Другие могут рассказать о том же, но по-своему. Мне кажется, это (такие повторы)~-- не страшно.} (<<дома образцового содержания>>)}

Нашим замечательным Управдомом долгие годы был Иван Максимович Калиш. Сохранилась фотография, на которой изображена скамейка в садике нашего Дома, а на ней мы с папой, Иван Максимович и А.И. Кузнецов. Подобные фотографии есть и в других семьях, поскольку И.М. Калиш со многими жильцами дома был в приятельских и даже дружеских отношениях. Потом его место заняла Александра Ниитична Давыдова, которая уже в 50-е годы поселилась в 6-ом подъезде нашего Дома.

Все аварийные ситуации в Доме успешно побеждали свой мастер на все руки (слесарь-сантехник, электрик и проч.) N.N. Барский и его помощница Таня N. Если не ошибаюсь, в конце сороковых годов дочь Барского вышла замуж за старшего сына полковника Ианцова из 5-го подъезда. Был и свой столяр, который с семьей жил в подвале 5-го подъезда (за красным уголком).

Весьма авторитетной фигурой для нас, мальчишек Дома, был Главный дворник Дядя Павел. А был еще второй дворник N. Шапиро.

В Доме была оборудована собственная котельная для обогрева квартир зимой, а в ванных вода нагревалась с помощью газовой горелки фирмы <<Юнкерс>>, что казалось забавным для детей войны. Некоторые жильцы у себя в квартирах сделали отвод горячей воды из ванной на кухню (по тем временам~-- чудеса да и только!). кто ыбл Главным истопником~-- не помню, но у него было два сына: Иелам и  Мареф. Вход в котельную был с заднего двора, а окна из нее выходили в подвал под вторым подъездом. Как-то летом, уже в конце 50-х или в начале 60-х, из этого подвала появилась перепачканная мордашка очаровательной девицы лет трех, которая доверительно обратилась ко мне с жалобой на жизнь: <<Я один, совсем один... мне так грустно...>> (детей во дворе~-- никого; это была дочка одной из сестер Гущиных, которая вышла замуж в Румынино и гостила у родных).

В Доме работали лифты (в 20-х годах это было событием). Вначале лифтеры круглосуточно дежурили в каждом подъезде. Затем, по два лифтера обслуживали все лифты. Их штаб-квартира располагалась в сторожке, а она~-- в подворотне.

Внутридомный Детсад (подробнее см. выше~-- воспоминания О.А. Гриневского) вначале размещался во 2-м подъезде, а затем  расширился за счет квартир первого этажа 3-го (окна на улицу) и 4-го (окна во двор) подъездов. Для ребят всех возрастов организовывались экскурсии по Москве.

В Доме еще до войны работал стол продовольственных заказов. Для постоянных клиентов в стену подворотни был встроен блок из двадцати с лишним сейфов-ячеек, в которые загружали готовые заказы, а каждый из клиентов имел ключи от своей ячейки. Привозили заказы но открытом пикапе. Мы помогали его разгружать; за это нас катали по двору (по улице нельзя!).

Пункт приема в стирку и выдачи чистого белья был организован в подвале 1-го подъезда, а затем перекочевал в подвал под 7-ым подъездом. В этом подвалев разные времена работала столовая (был и буфет).

собственная снеготаялка располагалась около ворот заднего двора, недалеко от котельной. Мусоросборник и мусоросжигалка также располагались на заднем дворе (около окон 3-го подъезда).

Особое место занимал и особую роль в жизни Дома играл Красный уголок, расположенный в подвале 5-го подъезда. В нем занимались самыми разнообразными делами и развлекались:

\begin{itemize}
\item проводились собрания жильцов, занятия Ликбеза, выступалли агитаторы в период проведения первых послевоенных выборов, готовились стенгазеты;
\item работали различные кружки: радиолюбителей, хоровой, драматический, моего сына учили даже играть на фортепиано, правда, не очень успешно;
\item иногда <<крутили>> кино, проходили концерты художественной самодеятельности;
\item работала библиотека;
\item можно было поиграть в биллиард, в настольные игры;
\item и многое-многое другое.
\end{itemize}

Это все были замечательные особенности Дома у Красных ворот.

Были и свои казусы:
\begin{itemize}
\item маленькие, в некоторых квартирах~-- крохотные уборные;
\item кухни трех размеров: крохотные, небольшие и побольше (до 8-9 метров);
\item во многих квартирах были предусмотренны шкафы-холодильники (с окошком на улицу), во многих~-- нет;
\item в одних квартирах были балконы, в других~-- нет, например на 3-м этаже 3-го подъезда: в кв. 35 был огромный балкон, а в кв. 34 и 36 балконов не было;
\item в <<левых>> (при подъеме по лестнице~-- налево) квартирах 3-го подъезда были затемненные комнаты и кухни (виноват странный выступ~-- квартиры 2-го подъезда). 
\end{itemize}

Я уже не говорю о том, как Дом образцового содержания посттрадал за свои привелегии, сколько людей~-- не только Наркоминдельцев~-- погибло в сталинских застенках.
