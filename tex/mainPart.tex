\chapter{Дворовые компании}

Многие данные и фоотографии предоставлены мне Т. В. Меньшиковой. Ей же благодарен за добрые советы.

Мне поручили написать о дворовых компанях, что я и постараюсь сделать с учетом <<собирательных>> свойств нашего Дома за период с тридцатых до шестидесятых годов. Аналог этого книжного варианта, естественно, будет в Альбоме.

Нельзя не сказать, что среди первопоселенцев нашего дома было очень много замечательных людей, начиная с Народного Комиссара иностранных дел М. М. Литвинова, его заместителей, руководителей подразделений, Чрезвычайных и полномочных посло того времени.

Сейчас, через 80 с лишним лет после заселения нашего дома, трудно сказать насколько были дружны между собой первопоселенцы, были ли дворовые компании у их старших детей (до 1920 года рождения), привлекались ли в эти компании ребята из других домов, что было характерно для последующих поколений, когда компании формировались из ребят окрестных дворов, как правило, в возрастном диапазоне 2-3 года с вкраплениями до 5 лет.

Собирательная особенность нашего двора наметилась еще до Войны, у ребят около 1924 года рождения, в основном выпускников школ 1941 года. К ним относятся: Бабенко И. Я., Багун Ю. (воевал), Варзар Сева (прошел всю войну), Дивильковский Юра (погиб), Клейн Наташа (воевала), Короткин Жора (погиб), Кунина Ляля (воевала), миникс Аба (погиб).

\newpage % ??
\newgeometry{left=10mm, right=10mm, top=5mm, bottom=0mm}

\begin{center}
    \includegraphics[width=0.5\textwidth]{inc/MalDev}
\end{center}
\begin{center}
    \footnotesize{
    \textbf {Мальчикам и девочкам 1940-х} \\
    (1924 года рождения)
    }
\end{center}

\scriptsize{
\begin{multicols}{2}
    
    \noindent
    К ребятам нашего двора \\
    Пришла не лучшая пора. \\
    \vfill
    \noindent
    Груз годов ведь с плеч не скинешь,\\
    И, куда не поглядишь,\\
    Всюду виден, вроде, фииниш,\\
    И осталось, вроде, шиш! \\
    \vfill
    \noindent
    Только нам, знать, срок не вышел,\\
    Только порох, видно, есть,\\
    Коли вновь мы, братцы, здесь,\\
    Коль мы снова тут, под крышей\\
    Дома номер два дробь шесть.\\
    \vfill
    \noindent
    Откуда шли, шпана дворовая,\\
    Когда пришла пора суровая\\
    Расстаться, шумною толпой\\
    В жизнь, как на праздник~--пей да пой!,--\\
    Ушли веселую шурьбой.\\
    Да каждый со своей судьбой.\\
    \vfill
    \noindent
    Пошли, в цепочку растянувшись,\\
    Один, едва начав, споткнувшись,\\
    Далече не успел уйти\\
    По незавдному пути;\\
    Другой, гляди, взлетел высоко,\\
    А третий ускакал далеко...\\
    \vfill   
    \noindent 
    А мир для всех куда как шире;\\
    В иных домах на нашец жизни ткань:\\
    То~-- шелк с парчей, то~-- просто дрянь!\\
    То планов дрезких грамадье,\\
    То только старое тряпье.\\
    \vfill
    \noindent
    Но жизнь~-- от Бога и надолго.\\
    И не забыть бы, братцы, долга\\
    За нами~-- тем, кто нас родил,\\
    Да тем, кто раньше уходил.\\
    Но не в прекрасную страну,\\
    А на проклятую войну.\\
    \vfill
    \noindent
    Утащила злая баба,\\
    Уложила в поле сдуру\\
    Твоео братишку Абу,\\
    Моего братана Юру,\\
    Сколько было их~-- не счесть!\\
    Всех помянем благородно,\\
    Отдадим посмертно честь!\\
    \vfill
    \noindent
    С непкурытой головою,\\
    То ли лысой, то ль седою,\\
    Постоим да помолчим.\\
    А потом уж прокричим\\
    Из последних сил, что есть,\\
    Славу дому два дробь шесть!\\
\end{multicols}
\vspace*{-5mm}
\raggedleft{С. И. Д. (Дивильковский С. И.), 2010 г.

}
}


\restoregeometry
