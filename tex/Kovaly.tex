%% Преамбула TeX-файла

% 1. Стиль и язык
\documentclass[utf8x, 12pt]{G7-32a} % Стиль (по умолчанию будет 14pt)

% Остальные стандартные настройки убраны в preamble-std.tex
\include{preamble-std}



\begin{document}



\frontmatter % выключает нумерацию ВСЕГО; здесь начинаются ненумерованные главы: реферат, введение, глоссарий, сокращения и прочее

% Команды \breakingbeforechapters и \nonbreakingbeforechapters
% управляют разрывом страницы перед главами.
% По-умолчанию страница разрывается.

% \nobreakingbeforechapters
% \breakingbeforechapters
\thispagestyle{empty} 

\begin{center}
    Глубокоуважаемый Борис Иосифович!
\end{center}

Как-то Юра Райский (7-й подъезд нашего Дома) по-хорошему назвал меня Восторженным идиотом. И вот лет через 65 я вновь выступаю в этом качестве. Ваша <<Харизма>>, правда, не только поразила меня глубиной, широтой, высотой мысли, великолепным построением и языком, стилем, но и обнадежила, поскольку я во многом разобрался; у меня появились даже некоторые соображения (правда~-- местного значения, для себя).

Однажды Резерфорд, обходя вечером свою лабораторию, спросил у сотрудника, что тот делает. <<Работаю>>. <<А что вы делали утром?>> <<Работал>>. <<Когда же вы думаете?>> удивился <<Крокодил>>. Так вот, я всю сознательную жизнь (а была и бессознательная) работал до одурения, до абсурда. В результате, я так и не научился широко, глубоко, свободно мыслить. Тем более я признателен Вам за <<Харизму>> и добрые слова на титуле.

Но есть еще и тайная цель этого послания. Таня Меньшикова продолжает надеяться подготовить сборник воспоминаний о Доме у Красных ворот и делает для этого все, что может. Пока есть только три большие статьи: О.А.~Гриневского, Э. И. Певзнер и моя, одна поменьше С. Л. Корчиковой (Клейн) и других, очень много фотографий. Кроме того, есть несколько выпусков <<Вундеркинда>>. Но всего этого, конечно, недостаточно. А ведь есть, насколько я знаю, превосходно пишущая гвардия ССД: И. Петров, З. Литвинова, Л. Фролова, Т. Зайцева (кто еще?).

Было бы очень здорово, если бы Вы, с Вашим авторитетом и Вашим опытом согласились активизировать и контролировать работу над воспоминаниям. Но главное~-- чтобы Вы воспользовались своим литературным даром во славу нашего Дома и ССД.

Вторая проблема~-- подготовка Торжественного альбома (сам альбом уже преобрели), в который должны войти компактные тексты (иногда таблицы) и фотографии с подписями по разделам, посвященным ветеранам войны и труда, репрессированным в 30-е~-- 40-е годы, погибшим на фронтах ВОВ, а также семьям (поквартирно) старожилов нашего Дома и дворовым компаниям различных поколений.

Для того, чтобы наглядно продемонстрировать свои представления о вариантах решения этих проблем, позвольте прислать\footnote{Ваше согласие и электронный адресс прошу прислать на имя моего сына Владимира Марковича: minix@rambler.ru для меня} Вам на обсуждения два моих материала: <<О людях дома у Красных ворот>> и <<Дворовые компании>>.

Жду Ваших добрых советов и реекомендаций на будущее, поскольку исправления можно оперативно внести на носители (так мы и поступим), но не в <<живые>> книжки.

Простите за многословие (по нашему~-- плеоназм).

Всего самого доброго.
 
\indent
 
Ваш Марк Миникс
 
\mainmatter % это включает нумерацию глав и секций в документе ниже



\backmatter %% Здесь заканчивается нумерованная часть документа и начинаются ссылки и
            %% заключение



\appendix   % Тут идут приложения



\end{document}

%%% Local Variables:
%%% mode: latex
%%% TeX-master: t
%%% End:
