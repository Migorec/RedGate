%% Преамбула TeX-файла

% 1. Стиль и язык
\documentclass[utf8x, 12pt, a5paper]{G7-32a} % Стиль (по умолчанию будет 14pt)

% Остальные стандартные настройки убраны в preamble-std.tex
\include{preamble-std}



\begin{document}



\frontmatter % выключает нумерацию ВСЕГО; здесь начинаются ненумерованные главы: реферат, введение, глоссарий, сокращения и прочее

% Команды \breakingbeforechapters и \nonbreakingbeforechapters
% управляют разрывом страницы перед главами.
% По-умолчанию страница разрывается.

% \nobreakingbeforechapters
% \breakingbeforechapters
\thispagestyle{empty} 


\noindent

\begin{center}

{\itshape
Светлая благодарная память \\ замечетельной Елене Васильевне,\\
которая в самые трудные\\ для моей мамы годы \\ <<незаметно>>, но очень ощутимо \\
оказывала ей так необходимую \\ моральную поддержку.\\
Низкий им, нашим мамам, поклон.

}

\end{center}
\indent

\indent

\noindent
Сентябрь 2012г.

 
\newpage

\noindent

\begin{center}

{\itshape
Светлая благодарная память \\ замечетельной Елене Васильевне,\\
которая в самые трудные\\ для моей мамы годы \\ <<незаметно>>, но очень ощутимо \\
оказывала ей так необходимую \\ моральную поддержку.\\
Низкий им, нашим мамам, поклон.

}

\end{center}
\indent

\indent

\noindent
Сентябрь 2012г.
 
\mainmatter % это включает нумерацию глав и секций в документе ниже



\backmatter %% Здесь заканчивается нумерованная часть документа и начинаются ссылки и
            %% заключение



\appendix   % Тут идут приложения



\end{document}

%%% Local Variables:
%%% mode: latex
%%% TeX-master: t
%%% End:
