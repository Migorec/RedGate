\singlespacing

\chapter{\protect\footnote{Все Приложения даны в редакции авторов воспоминаний.}}

\section*{Из <<Вундеркинда>> №1. 1947г.}

Первая сессия Верховного совета ССД была открыта 31 марта 1947г. В Большом Дворце Красного Уголка. Сессию открыл старейший депутат т. Прейс.

Выбрав Председателя, депутаты выслушали отчет Председателя Центральной избирательной комиссии т.  Горностаева и приступили к выборам Президента.

Голосованием было установлено, что на пост Президента ССД избран тов. Дивильковский.

Затем т. Прейс зачитал список Министров, а те, в свою очередь, выбрали Председателя совета Министров: тов. Гриневского.

Утвержден также Гимн ССД (поется на мотив из <<Девушки моей мечты>>.):

\vfill

{\itshape

Славься, славься Государство наше,

Самая прекрасная страна.

Ты всех стран сильнее, могучее и краше

Круглый год в тебе стоит весна!

\vfill

Славься, славься ССД во веки~--

Демократия в тебе сильна.

Здесь живут свободно человеки,

Круглый год в тебе стоит весна!!

\vfill

В ССД вся власть в руках народа,

Много в нем закусок и вина.

В ССД господствует свобода,

В ССД всегда стоит весна!!!

}

\onehalfspacing

\newgeometry{top=7mm,left=5mm,right=5mm,bottom=15mm, twoside}

\section*{Из <<Вундеркинда>> №2. 1947г.}

\noindent
\includegraphics[width=\textwidth]{inc/Vynd/Vynd001}

\newpage

\noindent
\includegraphics[width=\textwidth]{inc/Vynd/Vynd002}

\noindent
\includegraphics[width=\textwidth]{inc/Vynd/Vynd003}

\newpage


\noindent
\includegraphics[width=\textwidth]{inc/Vynd/Vynd004}

\noindent
\includegraphics[width=\textwidth]{inc/Vynd/Vynd005}

\noindent
\includegraphics[width=\textwidth]{inc/Vynd/Vynd006}

\noindent
\includegraphics[width=\textwidth]{inc/Vynd/Vynd007}

\noindent
\includegraphics[width=\textwidth]{inc/Vynd/Vynd008}

\newpage

\noindent
\includegraphics[width=\textwidth]{inc/Vynd/Vynd009}

\noindent
\includegraphics[width=\textwidth]{inc/Vynd/Vynd010}

\noindent
\includegraphics[width=\textwidth]{inc/Vynd/Vynd011}

\vfill

\begin{figure}[h!]
    \begin{minipage}{100mm}
    \includegraphics[width=100mm]{inc/49/1}
    \textit{\footnotesize{На занятиях радиокружка в нашем Красном уголке.}}
    \end{minipage}
\end{figure}

\newpage


\noindent
\includegraphics[width=\textwidth]{inc/Vynd/Vynd012}

\newpage

\noindent
\includegraphics[width=\textwidth]{inc/Vynd/Vynd013}

\section*{Из <<Вундеркинда>> юбилейного. 1987 год.}

\noindent
\includegraphics[width=\textwidth]{inc/Vynd/Vynd014}

\noindent
\includegraphics[width=\textwidth]{inc/Vynd/Vynd016a}

\noindent
\includegraphics[width=\textwidth]{inc/Vynd/Vynd017a}

\noindent
\includegraphics[width=\textwidth]{inc/Vynd/Vynd018}

\noindent
\includegraphics[width=\textwidth]{inc/Vynd/Vynd019a}

\noindent
\includegraphics[width=\textwidth]{inc/Vynd/Vynd020a}

\vfill

\begin{figure}[h!]
    \begin{minipage}{100mm}
    \includegraphics[width=100mm]{inc/57/1}
    \textit{\footnotesize{Начало XX века. Красные ворота, а за ними Запасный дворец, который после реконструкции превратился в здание МПС (сегодня РЖД).}}
    \end{minipage}
\end{figure}

\newpage


\noindent
\includegraphics[width=\textwidth]{inc/Vynd/Vynd021}

\noindent
\includegraphics[width=\textwidth]{inc/Vynd/Vynd022}

\noindent
\includegraphics[width=\textwidth]{inc/Vynd/Vynd023}

\noindent
\includegraphics[width=\textwidth]{inc/Vynd/Vynd024}

\noindent
\includegraphics[width=\textwidth]{inc/Vynd/Vynd025}

\noindent
\includegraphics[width=\textwidth]{inc/Vynd/Vynd026}

\newpage

\vspace*{80pt}

\noindent
\includegraphics[width=\textwidth]{inc/Vynd/Vynd027}

\newpage

\vspace*{120pt}

\noindent
\includegraphics[width=\textwidth]{inc/Vynd/Vynd028}

\restoregeometry

